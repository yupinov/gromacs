\documentclass[11pt]{beamer}
\usetheme{Berlin}
\usepackage[utf8]{inputenc}
\usepackage{amsmath}
\usepackage{amsfonts}
\usepackage{amssymb}
\usepackage{placeins}
\usepackage{graphicx}
\usepackage{tikz}
\author{Aleksei Iupinov}
\title{Particle mesh Ewald on a GPU}
%\setbeamercovered{transparent} 
%\setbeamertemplate{navigation symbols}{} 
%\logo{} 
\institute{KTH Royal Institute of Technology} 
%\date{} 
%\subject{} 
\begin{document}

\begin{frame}
\titlepage
\end{frame}

\begin{frame}
\tableofcontents
\end{frame}

\section{Introduction}
\begin{frame}{Molecular dynamics simulation performance}
\begin{itemize}
\item modelling large organic molecules (hundreds of thousands atoms)
\item scaling performance is important (timesteps of $10^{-15}$ seconds, processes of $10^{-6}$ -- $10^{-3}$ seconds) 
\item the balance shifting towards GPU/accelerator hardware (core speed limited, increasing number of cores instead)
\item most computational time -- electrostatic interactions
\item PME used for computing long-range component
\end{itemize}
\end{frame}

\section{Problem}
\begin{frame}{Electrostatic interactions}
\begin{itemize}
\item particles coordinates $r_1 .. r_N$ and charges $q_1 .. q_N$ known, forces acting on particles to be computed 
\item electrostatic potential:
\[E = \frac{1}{4 \pi \varepsilon_0}\sum\limits_{i < j}\frac{q_i q_j}{\lvert r_i-r_j\rvert}\]
% force derivation
\item large $N$ and computational effort $O(N^2) \implies$ low performance!
\end{itemize}
\end{frame}

\section{Method}
\begin{frame}{Ewald summation}
\begin{itemize}
\item interaction dependency is $\frac{1}{r}$ - not converging easily
% graph of decompose
\item decomposition into direct and reciprocal space components
% formulas
\item the direct space component converges before a certain cut-off
\item the reciprocal component converges well in the Fourier space
\end{itemize}
\end{frame}

\begin{frame}{Ewald summation assumptions}
\begin{itemize}
\item Fourier component $\implies$ periodic boundary conditions (system is a unit cell tiled in all directions)
\item charge neutrality of the unit cell is needed for convergence
%\item method originally used in crystallography
%lattice picture
%formulas
\end{itemize}
\end{frame}

\begin{frame}{PME}
\end{frame}

\section{Implementation}
\begin{frame}{Implementation}
\end{frame}

\section{Results}
\begin{frame}{Results}
\end{frame}

\section{Conclusion}
\begin{frame}{Conclusion}
\end{frame}

\begin{frame}[plain]
      Thank you for listening!
\end{frame}

%references











\iffalse
\includegraphics[scale=0.15]{electrical.png} 
\[\dot{y}_L(t) = f_L(y_L, y_A, t); \;\;y_L(t_0)=y_{L,0}\]
\[\dot{y}_A(t) = f_A(y_L, y_A, t); \;\;y_A(t_0)=y_{A,0}\]

\begin{itemize}
\item the current implementation of the MPRK method is not suitable for our neuron model due to low performance
\item the method was developed for modeling latent electrical circuits, so the assumption by the method's authors is a very small number of active components (not in our case - 22 out of 40)
\item likely both our systems are stiff, so dynamic reassigning of latent/active roles based on error estimations would be interesting to study - not supported by the framework
\item parallelization of active steps during latent step possible - better run on dozens/hundreds of cores to matter
\fi


\end{document}