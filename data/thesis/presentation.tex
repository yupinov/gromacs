\documentclass[11pt]{beamer}
\usetheme{Berlin}
\usepackage[utf8]{inputenc}
\usepackage{amsmath}
\usepackage{amsfonts}
\usepackage{amssymb}
\usepackage{placeins}
\usepackage{graphicx}
\usepackage{tikz}

\DeclareMathOperator\erf{erf}
\DeclareMathOperator\erfc{erfc}


\author[Aleksei Iupinov]{Aleksei Iupinov\\{\footnotesize Supervised by: Berk Hess}}
%\author{Aleksei Iupinov}
\title{Master thesis project: \\Particle mesh Ewald on a GPU}
\usepackage{subcaption}
%\setbeamercovered{transparent} 
\setbeamertemplate{navigation symbols}{} 
%\logo{} 
\institute{KTH Royal Institute of Technology} 
%\date{} 
%\subject{} 
\begin{document}
%\setbeamertemplate{caption}{\raggedright\insertcaption\par}
\captionsetup[figure]{labelformat=empty}

\begin{frame}
\titlepage
\end{frame}

\begin{frame}
\tableofcontents
\end{frame}

\section{Introduction}
\begin{frame}{Molecular dynamics simulation performance}
\begin{itemize}
\item modelling large organic molecules (hundreds of thousands atoms)
\item scaling performance is important (timesteps of $10^{-15}$ seconds, processes of $10^{-6}$ -- $10^{-3}$ seconds) 
\item the balance shifting towards GPU/accelerator hardware (core speed limited, increasing number of cores instead)
\item most computational time taken by electrostatic interactions
\item PME (particle mesh Ewald) used for computing so-called long-range component
\end{itemize}
\end{frame}

\section{Problem}
\begin{frame}{Electrostatic interactions}
\begin{itemize}
\item particles coordinates $\bold{r}_1, \dotsc, \bold{r}_N$ and charges $q_1, \dotsc, q_N$ known, forces $\bold{f}_1, \dotsc, \bold{f}_N$ acting on particles to be computed 
\item electrostatic potential:
\[E(\bold{r}_1, \dotsc, \bold{r}_N) = \frac{1}{4 \pi \varepsilon_0}\sum\limits_{i < j}\frac{q_i q_j}{\lvert \bold{r}_i-\bold{r}_j\rvert}\]
\item forces are derived analytically:
\[\bold{f_i} = -\frac{\partial E}{\partial \bold{r_i}} \]
\item large $N$ and computational effort $O(N^2) \implies$ slow computation!
\end{itemize}
\end{frame}

\section{Method}
\begin{frame}{Ewald sum}
\begin{itemize}

\item interaction dependency is $\frac{1}{\lvert \bold{r}\rvert }$
\item Ewald sum: decomposition into direct and reciprocal space components, based on Gaussian error function
% formulas
\item the direct space component converges within a certain cut-off
\item the reciprocal component converges in the Fourier space
\end{itemize}
%\noindent\makebox[\textwidth][c]

\end{frame}

\begin{frame}{Ewald sum}
\[E = E_{dir} + E_{rec} + E_{corr}\]
\[
E_{dir} = \frac{1}{2}\sum\limits_{\bold{n\in \mathbb{Z}^3}}^*\sum\limits_{i=1}^N\sum\limits_{j=1}^N\frac{q_i q_j \erfc(\beta\lvert\bold{r}_i - \bold{r}_j + \bold{n} \rvert)}{\lvert\bold{r}_i - \bold{r}_j + \bold{n} \rvert}
\]
\[
E_{rec}=\frac{1}{2\pi V}\sum\limits_{\bold{m} \in \{\mathbb{Z}^3 \setminus \mathbb{0}\}}\frac{\exp(-\frac{\pi^2\bold{m}^2}{\beta^2})}{\bold{m}^2}S(\bold{m})S(\bold{-m})
\]
\[
 S(\bold{m}) = 
\sum\limits_{j=1}^Nq_j \exp(2\pi i\bold{m}\cdot\bold{r}_j)
\]
\[E_{corr} = -\frac{1}{2}\sum\limits_{(i,j)\in M}\frac{q_i q_j \erf(\beta\lvert\bold{r}_i - \bold{r}_j\rvert)}{\lvert\bold{r}_i - \bold{r}_j\rvert} - \frac{\beta}{\sqrt{\pi}}\sum\limits_{i=1}^N q_i^2\]

\begin{itemize}
\item Fourier component $\implies$ periodic boundary conditions (system is a unit cell tiled in all directions)
\item charge neutrality of the unit cell is needed for convergence
\item additional correction component (bonded interactions split from direct sum)
\item can be implemented in $O(N^\frac{3}{2})$
%\item manipulating the cut-off can change the balance

%\item method originally used in crystallography
%lattice picture
%formulas
\end{itemize}
\end{frame}

\begin{frame}{Particle mesh Ewald}
\begin{itemize}
%picture of charges
\item a way to approximate exponentials in the reciprocal sum 
\item interpolation onto a discrete 3D grid
\item smooth PME: B-Spline interpolation, the potential function is smooth $\implies$ simple analytical derivation of forces
\item can be implemented in $O(N \log(N))$
% formulas
\end{itemize}
\end{frame}

\begin{frame}{PME stages}
\begin{enumerate}
\item calculate the B-spline interpolation values for all particles
\item spread the particle charges on a discrete 3D grid
\item perform real-to-complex 3D FFT of the grid
\item calculate the reciprocal energy contribution(equation \eqref{Erecfull}), transform the grid
\item perform complex-to-real 3D FFT of the grid
\item gather the particle forces from the grid
\end{enumerate}
\end{frame}

%pictures?

\section{Implementation}
\begin{frame}{Implementation}
\begin{itemize}
\item open source MD software GROMACS 5.1
\item short-range particle pair interactions computed either on CPU or GPU (CUDA/OpenCL) 
\item PME only implemented for CPU
%\item PME GPU implementation can mean: more performance, more load balancing, more GPU hardware utilisation, less CPU bottleneck 
%mention cutoff
\end{itemize}
\end{frame}

\begin{frame}{Potential benefits of PME on GPU}
% should be moved
\begin{itemize}
\item more performance
\item more flexibility, possibilities for load balancing 
\item nearly all computation can be offloaded onto GPUs
\item avoiding CPU bottleneck on machines with lots of GPU power  
\end{itemize}
\end{frame}

\begin{frame}{MD time step in GROMACS}
\begin{figure}
    \centering
    \includegraphics[width=1\textwidth]{pics/mdstep-orig.png}
    \caption{Timeline of MD step in a single-process GROMACS simulation}
    \label{fig:step-orig}
\end{figure}
\FloatBarrier
\end{frame}

\begin{frame}{MD time step in GROMACS, modified for PME GPU}
\begin{figure}
    \centering
    \includegraphics[width=1\textwidth]{pics/mdstep-gpu.png}
    \caption{Timeline of a modified MD step with PME GPU}
\end{figure}
\FloatBarrier
\end{frame}

\begin{frame}{Calculating the spline parameters}
\begin{itemize}
\item...
\end{itemize}
\end{frame}

\begin{frame}{Spreading the charges}
\begin{itemize}
\item a large part of the runtime
\item charges accumulated onto the grid in parallel $\implies$ clashes of atomic operations
\item a single thread works on one line of contributions
\item a warp works exactly on 2 particles for order of 4
%..order....
\end{itemize}
\end{frame}

\begin{frame}{3D FFT}
\begin{itemize}
\item happens twice (real-to-complex before solving, complex-to-real after solving)
\item implemented as a call to CUDA cuFFT library
\item with multiple PME processes requires all-to-all communication of the decomposed grid
\item for this reason an option of using the CPU code (MPI + calls to FFTW library) is retained
% overlap
\end{itemize}
\end{frame}

\begin{frame}{Solving in Fourier space}
\begin{itemize}
\item a small compute-bound kernel
\item performance dependent on the grid size
\item a single thread works on one grid cell
\item block works on one/several grid lines
\item energy reduction in shared memory per block, then incrementing a global result atomically
\end{itemize}
\end{frame}


\begin{frame}{FFT}
\end{frame}

\begin{frame}{weird FFT}
\end{frame}

\section{Results}
\begin{frame}{Results}
\begin{itemize}
\item implemented for a single GPU
\item supports GPU and CPU FFT, which is the main parallelization barrier 
\item only interpolation order of 4 (again, not a big limitation, the core is there)
%\item works in a single PME/PP process
%\item or as a single PME process with several PP ranks
\item load balancing works
\item integrated soon, accessible here
% more about decomposition
% more about tuning
% more about B-splines
% references?
\end{itemize}
\end{frame}



\begin{frame}{Single process CPU/GPU comparison, 134201 atoms}
\begin{itemize}
\item short-range computed on GPU (NVIDIA GTX Titan X)
\item PME computed either on a CPU (Intel Core i7-5960X) or on the same GPU
% threads varied
\end{itemize}
\FloatBarrier
\begin{figure} [h!]
    \centering
    \includegraphics[width=0.8\textwidth]{pics/CPU_GPU_ADH_SINGLE.png}
\end{figure}
% superiority over weak CPUs
\FloatBarrier
\end{frame}

\begin{frame}{2 processes (1 PP / 1 PME) CPU/GPU comparison}
\begin{itemize}
\item short-range computed on GPU (NVIDIA GTX Titan X)
\item PME computed either on a CPU (Intel Core i7-5960X) or on the second GPU (NVIDIA Quadro M6000)
\end{itemize}
\FloatBarrier
% threads varied
\begin{figure} [h!]
    \centering
    \includegraphics[width=0.8\textwidth]{pics/CPU_GPU_ADH.png}
    \label{fig:sepGPUNEW}
\end{figure}
\FloatBarrier
\end{frame}

\section{Conclusion}

\begin{frame}{Conclusion}
\begin{itemize}
\item the single GPU implementation provides a basis for further tuning
\item supports a separate PME process with direct space decomposed over multiple PP processes, already viable for machines with several NVIDIA GPUs 
\item should be easily extensible for scaling over multiple GPUs
(compatible with the CPU FFT code supporting the Fourier grid decomposition)
\item tshould also be easily extensible for interpolation orders other than 4
(higher orders $\implies$ coarser grid $\implies$ less communication)

\end{itemize}
\end{frame}

\begin{frame}[plain]
      Thank you for listening!
\end{frame}

%references

%PP, NB, other terms

% where does the grid size come from?x

\end{document}