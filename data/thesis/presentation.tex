\documentclass[11pt]{beamer}
\usetheme{Berlin}
\usepackage[utf8]{inputenc}
\usepackage{amsmath}
\usepackage{amsfonts}
\usepackage{amssymb}
\usepackage{placeins}
\usepackage{graphicx}
\usepackage{tikz}
\author{Aleksei Iupinov}
\title{Particle mesh Ewald on a GPU}
%\setbeamercovered{transparent} 
%\setbeamertemplate{navigation symbols}{} 
%\logo{} 
\institute{KTH Royal Institute of Technology} 
%\date{} 
%\subject{} 
\begin{document}

\begin{frame}
\titlepage
\end{frame}

\begin{frame}
\tableofcontents
\end{frame}

\section{Introduction}
\begin{frame}{Molecular dynamics}

\begin{itemize}
\item modeling plenty of processes on plenty of time and space scales
\item simulations have to stay coherent, exchange data
\item we want a partitioned numerical method that accounts for this and is robust
\end{itemize}

\end{frame}

\section{Problem}
\begin{frame}{Electrostatic interactions}
\end{frame}

\section{Method}
\begin{frame}{Ewald sum and PME}
\end{frame}

\section{Implementation}
\begin{frame}{Implementation}
\end{frame}

\section{Results}
\begin{frame}{Results}
\end{frame}

\section{Conclusion}
\begin{frame}{Conclusion}
\end{frame}

\begin{frame}[plain]
      Thank you for listening!
\end{frame}

%references











\iffalse
\section{Introduction}
\begin{frame}{Modeling of biological systems}
\begin{itemize}
\item modeling plenty of processes on plenty of time and space scales
\item simulations have to stay coherent, exchange data
\item we want a partitioned numerical method that accounts for this and is robust
\end{itemize}

\end{frame}
\begin{frame}{Modeling of a neuron}
\begin{itemize}
\item neurons feature electrical and chemical processes
\item current injection $\Rightarrow$ change of chemical system's steady state
\item the electrical system is generally much faster than the chemical system
\item looks like a potentially stiff problem
\end{itemize}
\includegraphics[scale=0.15]{electrical.png} 
\includegraphics[scale=0.15]{chemical.png} 
%\picture 1 - calcium!
%\picture 2 - electric

\end{frame}

\section{Problem}
\begin{frame}{Framework}
\begin{itemize}
\item the neuron model implemented in Matlab by Jerker Nilsson during his master thesis work in 2014
\item he implemented and compared several numerical methods (RK4, Hines, BDF)
\item the scope of this project was to try another numerical method in this framework: multirate partitioned Runge–Kutta method of the 4th order
\end{itemize}
\end{frame}

\section{Method}
\begin{frame}{MPRK method}
\begin{itemize}
\item assumes that ODE system is clearly split into latent (stiff) and active (non-stiff) parts:
\[\dot{y}_L(t) = f_L(y_L, y_A, t); \;\;y_L(t_0)=y_{L,0}\]
\[\dot{y}_A(t) = f_A(y_L, y_A, t); \;\;y_A(t_0)=y_{A,0}\]
\item implicit RK4 method with large step for the stiff part
\item explicit RK4 method with small step for the non-stiff part
\item coupling at the end of each large step
\item in our case electric is latent, chemical is active
\end{itemize}
\end{frame}

\begin{frame}{Step size control}
\begin{itemize}
\item as the methods are of the 4th order, the 3rd order RK methods with same stage values are embedded for error estimation
\item step size coefficients are bounded to avoid zigzagging (abrupt changes) 
\end{itemize}
\[ h_{new} = h_{old}\cdot \min(1.5, \max(0.5, 0.9 \cdot \Big(\frac{\text{absolute tolerance}}{\text{error estimation}}\Big)^\frac{1}{4}))\]
\end{frame}

\section{Results}
\begin{frame}{A small stiff problem}
\begin{itemize}
\item the method has been implemented and tested on a small problem
\item 2 components, weak coupling
\item compared the solution error with Matlab's stiff ODE solvers ode23s/ode15s
\end{itemize}
\includegraphics[scale=0.25]{final1.png} 
\includegraphics[scale=0.21]{zoom.png} 
\end{frame}

\begin{frame}{Unsatisfactory performance}
\begin{itemize}
\item also compared the number of steps, function calls, running time with ode23s/ode15s
\item number of latent steps larger by order of 1
\item number of function calls for active part larger by order of 3
\item running time also larger by order of 3
\item same bad performance on the main problem (22 active + 18 latent components)
\end{itemize}
\end{frame}

\section{Conclusion}
\begin{frame}{Conclusion}
\begin{itemize}
\item the current implementation of the MPRK method is not suitable for our neuron model due to low performance
\item the method was developed for modeling latent electrical circuits, so the assumption by the method's authors is a very small number of active components (not in our case - 22 out of 40)
\item likely both our systems are stiff, so dynamic reassigning of latent/active roles based on error estimations would be interesting to study - not supported by the framework
\item parallelization of active steps during latent step possible - better run on dozens/hundreds of cores to matter
\end{itemize}

\end{frame}

\fi


\end{document}